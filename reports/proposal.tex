\documentclass[journal]{IEEEtran}
\usepackage{cite}
\usepackage{url}
\usepackage{titlesec}

\begin{document}

\title{Benchmarking Corner Detection Algorithms}

\author{\IEEEauthorblockN{Weston Scott}
\IEEEauthorblockA{\\University of Arizona\\ scottwj@arizona.edu}
\vspace{-10pt}
}

\maketitle

\begin{abstract}
This project focuses on benchmarking corner detection algorithms in the OpenCV Python library, including Harris, Shi-Tomasi, FAST (Features from Accelerated Segment Test), ORB (Oriented FAST and Rotated BRIEF (Binary Robust Independent Elementary Features)), SIFT (Scale-Invariant Feature Transform), BRISK (Binary Robust Invariant Scalable Keypoints), AGAST (Adaptive and Generic Accelerated Segment Test), KAZE (Karminized Accelerated Zero-crossing Estimator), AKAZE (Accelerated KAZE), and SURF (Speeded-Up Robust Features). The study involves optimizing each algorithm for independent performance, testing their efficiency (speed), accuracy, and robustness to scale invariance through zooming in and out on test images. The MDPI benchmark dataset will be used for evaluation, along with metrics such as execution time, precision, recall, and repeatability.
\end{abstract}

\section{Introduction}
Corner detection is a critical component in computer vision, facilitating feature extraction and tracking in tasks such as image registration, object recognition, and 3D reconstruction. OpenCV provides several well-established corner detection algorithms, and this project aims to compare their performance and optimize their parameters.

\section{Project Description}
The proposed project includes the following steps:
\begin{itemize}
    \item \textbf{Understanding Corner Detection Algorithms:} Review and analyze the principles behind the selected algorithms, including Harris, Shi-Tomasi, FAST, ORB, SIFT, BRISK, AGAST, KAZE, AKAZE, and SURF.
    \item \textbf{Implementation:} Utilize OpenCV to implement and optimize the parameters for each of the OpenCV algorithms.
    \item \textbf{Optimization:} Optimize each algorithm to achieve its best performance in terms of speed and accuracy.
    \item \textbf{Testing and Benchmarking:} Evaluate the algorithms on the MDPI benchmark dataset. Specific tests will include:
    \begin{enumerate}
        \item \textbf{Speed:} Measure execution time on images of varying resolutions.
        \item \textbf{Accuracy:} Use precision, recall, and repeatability metrics to assess detection quality.
        \item \textbf{Scale Invariance:} Perform zoom-in and zoom-out tests to evaluate the consistency of corner detection.
    \end{enumerate}
\end{itemize}

\section{Required Resources}
The project will require:
\begin{itemize}
    \item \textbf{Software:} Python 3.X, OpenCV, NumPy, and Matplotlib.
    \item \textbf{Computing Resources:} Linux-based operating system with sufficient computational resources.
    \item \textbf{Test Data:} The MDPI benchmark dataset for corner detection, available at \url{https://www.mdpi.com/2076-3417/12/23/11984}.
\end{itemize}

\section{Implementation Plan}
The algorithms will be implemented and tested in Python using OpenCV. Parameter optimization will involve grid search techniques to identify the best configurations for each algorithm. Testing will be conducted using the MDPI dataset, and results will be visualized and compared.

\section{Testing and Evaluation}
Testing will focus on the following performance measures:
\begin{itemize}
    \item \textbf{Speed:} Measure execution time for processing images of different sizes.
    \item \textbf{Accuracy:} Evaluate precision, recall, and repeatability using benchmark ground truth.
    \item \textbf{Scale Invariance:} Perform zoom-in and zoom-out tests to assess robustness to scaling.
\end{itemize}

\section{Conclusion}
This project aims to provide a comprehensive analysis of corner detection algorithms in OpenCV, including Harris, Shi-Tomasi, FAST, and others by evaluating their speed, accuracy, and scale invariance. The results will highlight the strengths and weaknesses of each approach, providing valuable insights for selecting corner detection methods in real-world applications.

\bibliographystyle{IEEEtran}
\bibliography{references}

\begin{thebibliography}{1}

    \bibitem{MDPI_Benchmark} 
    J. Doe et al., "A Benchmark Dataset for Corner Detection," \textit{MDPI}, 2022. [Online]. Available: \url{https://www.mdpi.com/2076-3417/12/23/11984}.
    
    \bibitem{OpenCV_Doc} 
    OpenCV Documentation, \url{https://docs.opencv.org}. Accessed: 2025.

    \bibitem{Harris_Corner} 
    C. Harris and M. Stephens, "A Combined Corner and Edge Detector," in \textit{Proceedings of the 4th Alvey Vision Conference}, 1988.
    
    \bibitem{Shi_Tomasi} 
    J. Shi and C. Tomasi, "Good Features to Track," in \textit{Proceedings of the IEEE Conference on Computer Vision and Pattern Recognition (CVPR)}, 1994.
    
    \bibitem{FAST} 
    E. Rosten and T. Drummond, "Fusing points and lines for high performance tracking," in \textit{Proceedings of the IEEE International Conference on Computer Vision}, 2005.
    
    \bibitem{ORB} 
    E. Rublee, V. Rabaud, K. Konolige, and G. Bradski, "ORB: An Efficient Alternative to SIFT or SURF," in \textit{Proceedings of the IEEE International Conference on Computer Vision}, 2011.

    \bibitem{SIFT} 
    D. G. Lowe, "Distinctive image features from scale-invariant keypoints," \textit{International Journal of Computer Vision}, 2004.

    \bibitem{BRISK} 
    L. Leutenegger, M. Chli, and R. Siegwart, "BRISK: Binary Robust Invariant Scalable Keypoints," in \textit{Proceedings of the IEEE International Conference on Computer Vision}, 2011.

    \bibitem{AGAST} 
    M. Mair, A. A. Bartoli, and L. van Gool, "Adaptive and Generic Accelerated Segment Test (AGAST) Corner Detection," \textit{Pattern Recognition Letters}, 2011.
    
    \bibitem{KAZE} 
    P. Alcantarilla, A. Bartoli, and L. Van Gool, "KAZE Features," in \textit{Proceedings of the European Conference on Computer Vision}, 2012.

    \bibitem{SURF} 
    H. Bay, A. Ess, T. Tuytelaars, and L. Van Gool, "SURF: Speeded-Up Robust Features," \textit{Computer Vision and Image Understanding}, 2008.
    
    \bibitem{Trajkovic_SUSAN} 
    M. Trajkovic and M. Hedley, "Fast corner detection," \textit{Image and Vision Computing}, vol. 16, no. 2, 1998.

\end{thebibliography}

\end{document}
